\RequirePackage{silence}
\WarningFilter{scrbook}{Usage of package `fancyhdr'}  % Hepthesis issues
\WarningFilter{scrbook}{Usage of package `tocbibind'} % Hepthesis issues
\WarningFilter{biblatex}{File }  % Bei Gelegenheit gegenprüfen!
\WarningFilter{tocbasic}{`tocbibind' redefinition of `\listoffigures'}
\WarningFilter{tocbasic}{`tocbibind' redefinition of `\listoftables'} 
\WarningFilter{chktex}{You should perhaps use `\min' instead.} % unfortunately -
\WarningFilter{chktex}{You should perhaps use `\max' instead.} % - doesn't work
\WarningFilter{latex}{Marginpar on page} % due to todonotes
\WarningFilter{LaTeX}{Command }
\documentclass{akdai}

% Language-Encoding und Font
\usepackage{polyglossia}        % Alternative zu Babel
\usepackage{csquotes}           % Required by Polyglossia
\setmainlanguage[variant=british]{english}
\setotherlanguage[babelshorthands=true]{german}

% Citation, Verweise
\usepackage[style=ieee,backend=biber]{biblatex}
\addbibresource{BibLaTex/zotero.bib}
\usepackage{verbatimbox}
\usepackage{fancyvrb}
\usepackage[style=base]{caption}
\usepackage{listings}           % refer to 'minted vs. texments vs. verbments'
\usepackage{xcolor}
\usepackage{hyperref}           % Verlinkungen im Dokument
\hypersetup{
 colorlinks,
 linkcolor={red!45!black},
 citecolor={blue!50!black},
 urlcolor={blue!80!black}
}
\usepackage{nameref}
\usepackage{booktabs}           % Tabellenpaket-keine vertikal und dicken Linien
\usepackage[inline,shortlabels]{enumitem}
\usepackage[acronym]{glossaries}		% \gls{<golssary-entry>}
\setacronymstyle{long-short}
\usepackage{cleveref}
\lstset{
  frame=single,
  columns=fullflexible,
  literate={-}{-}1
}

\lstnewenvironment{python}[1][]{
  \definecolor{commentsColor}{rgb}{0.497495, 0.497587, 0.497464}
  \definecolor{keywordsColor}{rgb}{0.000000, 0.000000, 0.635294}
  \definecolor{stringColor}{rgb}{0.558215, 0.000000, 0.135316}
  \lstset{
    float=h,
    backgroundcolor=\color{white},                % choose the background color; you must add \usepackage{color} or \usepackage{xcolor}
    basicstyle=\footnotesize,                     % the size of the fonts that are used for the code
    breakatwhitespace=false,                      % sets if automatic breaks should only happen at whitespace
    breaklines=true,                              % sets automatic line breaking
    captionpos=b,                                 % sets the caption-position to bottom
    commentstyle=\color{commentsColor}\textit,    % comment style
    deletekeywords={},                            % if you want to delete keywords from the given language
    escapeinside={\%*}{*)},                       % if you want to add LaTeX within your code
    extendedchars=true,                           % lets you use non-ASCII characters; for 8-bits encodings only, does not work with UTF-8
    frame=tb,	                   	                % adds a frame around the code
    keepspaces=true,                              % keeps spaces in text, useful for keeping indentation of code (possibly needs columns=flexible)
    keywordstyle=\color{keywordsColor}\bfseries,  % keyword style
    language=Python,                              % the language of the code (can be overrided per snippet)
    otherkeywords={},                             % if you want to add more keywords to the set
    numbers=left,                                 % where to put the line-numbers; possible values are (none, left, right)
    numbersep=5pt,                                % how far the line-numbers are from the code
    numberstyle=\tiny\color{commentsColor}\noncopynumber,       % the style that is used for the line-numbers
    rulecolor=\color{black},                      % if not set, the frame-color may be changed on line-breaks within not-black text (e.g. comments (green here))
    showspaces=false,                             % show spaces everywhere adding particular underscores; it overrides 'showstringspaces'
    showstringspaces=false,                       % underline spaces within strings only
    showtabs=false,                               % show tabs within strings adding particular underscores
    stepnumber=1,                                 % the step between two line-numbers. If it's 1, each line will be numbered
    stringstyle=\color{stringColor},              % string literal style
    tabsize=2,                                    % sets default tabsize to 2 spaces
    title=\lstname,                               % show the filename of files included with \lstinputlisting; also try caption instead of title
    columns=fullflexible,                         % Using fixed column width produces nice alignment. I think it's ugly, though -> fullflexible
    literate={-}{-}1
             {*}{*}1
             {\ }{{\copyablespace}}1 {\ \ }{{\copyablespaceTwo}}1,
    #1                                            % Optional arguments
    }  
  }{} 

% Fonts
\usepackage{fontspec}
\usepackage{unicode-math} % fontspec wird auch von unicode-math geladen?
%\setmainfont{Times New Roman} % set fontsize: 11p
%\setmainfont{TeX Gyre Pagella} % set fontsize: 12p
%\setmathfont[ItalicFont=*, BoldFont=*]{TeX Gyre Pagella Math}

% Math-Packages und Fonts
\usepackage{physics} % ||norm|| und |abs|
\usepackage{amsmath}
\usepackage{mathrsfs} % für \mathscr

% Theoreme und Definitionen
\usepackage{amsthm}
\theoremstyle{definition}
\newtheorem{definition}{Definition}[section]

% Formatierung
\usepackage{geometry}           % Paket für Zeilenabstand 
\usepackage{setspace}           % Paket für Zeilenabstand 
\usepackage{microtype}


% Subsections
\setcounter{secnumdepth}{5}
\setcounter{tocdepth}{5}

% Fileinsertion
\usepackage{pdfpages}           % Paket zum Einfügen von PDFs

% Sonstiges                     % Einfügen von Grafiken
\usepackage{graphicx}
\usepackage{subcaption}         % Replacement for subfig package which is broken
\graphicspath{ {./pictures/} }
\usepackage{xcolor}
\usepackage[normalem]{ulem}     % Unterstreichen von Text mit \uline
\usepackage[de-DE]{datetime2}   % Kalenderdaten in deutschem Format
\usepackage{blindtext}          % Repräsentativer als Lorem Ipsum
\usepackage{kantlipsum}         % Cooler als blindtext
\setlength{\marginparwidth}{2cm}
\usepackage[disable,textwidth=15mm]{todonotes} % Einfügen von Todos, Option: [disable]
\usepackage[shortcuts]{extdash} % \=/ for nonbreaking dash
\usepackage{url}
\crefname{const}{constraint}{constraints}
\usepackage{xifthen}
\usepackage{xargs}
\usepackage{array,tabularx}

%%%%% Workarounds %%%%%
\setlength{\marginparwidth}{2cm} % without: trouble with todonotes

\newcommand*\appendixmore{ % add title to Abstract
  \addsec{\appendixname}%
  \renewcommand{\thesubsection}{\Alph{subsection}}%
}

% list of listings title
\renewcommand*{\lstlistlistingname}{List of listings}

% Make Python-Listing line numbers uncopyable
\usepackage[space=true]{accsupp}
\newcommand{\noncopynumber}[1]{%
    \BeginAccSupp{method=escape,ActualText={}}%
    #1%
    \EndAccSupp{}%
}
% Fix spaces in Python listing (really necessary?)
\newcommand{\copyablespace}{\BeginAccSupp{method=hex,unicode,ActualText=00A0}\ \EndAccSupp{}}
\newcommand{\copyablespaceTwo}{\BeginAccSupp{method=hex, unicode, ActualText=00A000A0}\ \ \EndAccSupp{}}
\makeatletter
  \def\lst@Literatekey#1\@nil@{\let\lst@ifxliterate\lst@if
  \expandafter\def\expandafter\lst@literate\expandafter{\lst@literate#1}}
\makeatother

%%%%% /Workarounds %%%%%

\author{
	Malte Zietlow, Finn Wellershaus \\ 
	\\
	Nordakademie\\ 
	Köllner Chaussee 11\\ 
	25337 Elmshorn \\ 
	malte.zietlow@nordakademie.de \\
	finn.wellershaus@nordakademie.de
}
\makeglossaries{}             % makeglossaries <document-root ohne .tex> per cmd

\begin{document}

\title{DeepFakes: Overview of technical possibilties of digital imitation and detection}
\maketitle

\listoftodos{}            % Aktuelle ToDo-Notes

% Frontmatter
\begin{abstract}
    In this paper we give a theoretical overview of methods for replicating a
    generic content manipulation technique named \textit{DeppFake}. Our focus
    lies on providing a mathematical formulation and giving a technical demo of
    it.
\end{abstract}
% Mainmatter
\section{Introduction}
Some introductory words.

\subsection{Structure}
\subsection{Limitations}

\section{Background}\label{sect:background}

\subsection{DeepFakes and CheapFakes}
In this work, a distinction between so called CheapFakes and \textit{DeepFakes}
is made. On the one hand, CheapFakes are produced by using contemporary video 
and picture editing software. The name CheapFake should under no circumstance be
interpreted as an assessment of their effectiveness since they can still be used
to effectively fool users. An example of this usage might be a video of US House
Speaker Nancy Pelosi which was slowed down for her to appear drunk~\cite{Fichera.2019}.
On the other hand, \textit{DeepFakes} describe the results of machine-learning
algorithms, which are most often a form of or combination of \glspl{nn}.
The names simply refer to the different levels of sophistication of the different
approaches. Arguably, at some point, they might even merge together with editing 
software utilizing machine learning more and \textit{DeepFake} technology becoming
accessible and integrated into existing tooling.

\subsection{Flavors of DeepFakes}\label{subsubsect:deepfake-flavors}
Broadly, \textit{DeepFakes} can be sorted into two categories, as described by~\textcite{Mirsky.2020}:
There is a source \textit{s} and a target \textit{t}

\begin{description}
    \item[Face Reenacment] Here, the expression of \textit{s} is used to drive
    the expression of \textit{t}~\cite{Mirsky.2020}. The source is
    therefore also called driver \textit{d}.
    \item[Replacement] In this case, some part of \textit{t} is replaced
    by the corresponding part of \textit{s}. An example would be the swapping of
    \textit{t}'s face with the one from \textit{s}, preserving \textit{t}'s mimic.
\end{description}

\subsection{Generalizability of Generative Models for DeepFakes}\label{subsubsect:generalizability-deepfakes}
In the design of \glspl{nn} for the creation of \textit{DeepFakes}, one must choose
the range of targets to which it is applicable~\cite[cf.][]{Mirsky.2020}:
\begin{description}
    \item[one-to-one] after training, the approach is applicable only to a
    single target and source/driver;
    \item[many-to-one] after training, the approach is applicable only to a
    single target, but arbitrary sources/drivers;
    \item[many-to-many] after training, the approach is applicable both to
    arbitrary targets and sources/drivers.
\end{description}
As mentioned in \cref{subsect:limitations}, only \textbf{many-to-many} approaches
are presented in this paper because they are more generic than \textbf{one-to-one} or
\textbf{many-to-one}.

\subsection{Ethical Challenges}
The importance of \textit{DeepFake} detection increases in conjunction with the
ethical challenges which arise from the use of this technology. The ability to
create arbitrary fake images bears multiple risks for misuse. Firstly, there is
the aspect of misinformation and propaganda. For example by showing images or
videos of political candidates saying things they have not said and that can
incriminate them. Moreover, another aspect to take into consideration would be
pornography, since it is possible to change faces to other people's bodies,
which can lead to non-consensual pornography. Following these dimensions, there
are two types of attacks: to public or political figures and to private
individuals. The first category is defined by attackers having more resources at
hand for an attack, e.g.\ an organized action against a specific political
candidate. The second category, in contrast, is defined by the attackers' lack
of adequate resources, e.g.\ an angry ex-boyfriend. Even though the second
category might pose a real risk since not that much effort might be needed to
spread a basic amount of false information, in the end, the attacker's
capabilities are somewhat limited. However, changes in the ease of use and
availability of \textit{DeepFake} technologies can affect its uses towards
targeting private individuals.
\section{Creation of DeepFakes}\label{sect:creation-of-deepfakes}
Statistical models come in multiple classes with divergent properties.
One such discrimination is that between discriminative and generative models.
Generative models learn the statistical properties of the input domain~\cite[cf.][\nopp~651\psqq]{Goodfellow.2016}.
For example, they are able to generate unique images of human faces, based on
previously observed ones~\cite{Karras.2019}. 

\par
Generative models are thus the focus of techniques for creating DeepFakes.
For an introduction to the creation of DeepFakes, the scope is limited to
\glspl{rnn}, \glspl{ae} and \glspl{gan}.
\begin{figure}[h]
    \center{}
    \includegraphics[width=0.65\textwidth]{basic_nns2.pdf}
    \caption{Selection of generative models, adopted from~\cite{Mirsky.2020}}\label{fig:drain-parse-tree}
\end{figure}

\section{Detection of DeepFakes}
We will focus on the detection of DeepFakes, even though there is also some research in the area of
prevention.

\subsection{Artifact-Based Approach}
The algorithms for creating DeepFakes might introduce artifacts in the resulting images.
These artifacts can be used to discover DeepFakes and lay the foundation for this type of detection.

\subsection{Undirected Approaches}

\section{Conclusion}
There are different approaches to dealing with \textit{DeepFakes}.
We consider methods of detection to be only one part of a possible solution since
on their own it would possibly only lead to a race between creating and discovering
networks which are constantly improving each other.
Here we agree with the authors of the servey that out of band methods for 
signatures of multimedia content is required for a better solution.

Also we would dissagree with the idea \textit{DeepFakes} are particullary easy to create.
Even though there are multiple projects like FaceSwap which aim at making the
process as accessible as possible, there is still the requirement of a decently 
powered computer, time and access to the proper video/picture footage.\todo{Creation conclusion}

% Backmatter
\appendix
\section{Identity loss}
\begin{equation}\label{eq:identity-loss}
    \mathcal{L}_{id}(\hat{y}_g,y_t)=-\sum{y_t, \log(\hat{y}_g)}
\end{equation}
where:
\begin{conditions}
    \hat{y}_g   & \text{ is } & the predicted identity of the generated DeepFake \(I(x_g)=\hat{y}_g\)\\
    y_t         & \text{ is } & the correct identity of the target \(x_t\)
\end{conditions}

\section{Adversarial loss}
\begin{equation}\label{eq:adversarial-loss}
    \mathcal{L}_{adv}(y)= \max_{ED}\min_{D}
    \begin{cases}
        {\left(1-D(y)\right)}^2,& \text{if } y=D({x^\prime}_t)\\
        {\left(D(y)\right)}^2,& \text{if } y=D(x_g)
    \end{cases}
\end{equation}

where:
\begin{conditions}
    ED              & \text{ is } & the combination of \gls{en} and \gls{de} as discussed in \cref{sect:creation-of-deepfakes}\\
    D               & \text{ is } & the \gls{gan}-like discriminator as discussed \hyperref[sect:creation-of-deepfakes]{ibid.}\\
    {x^\prime}_t    & \text{ is } & a random image of the target identity\\
    x_g             & \text{ is } & the image generated by the \gls{ed}
\end{conditions}
%\chapter{General Todos}

% Bibliography and lists 
{\hypersetup{hidelinks}
    \printbibliography{}
    \printglossaries{}
    \listoffigures
    \listoftables
}


% ----------------------------- Acronyms -----------------------------
\newacronym{nn}{NN}{Neural Network}
\newacronym{cnn}{CNN}{Convolutional Neural Network}

% ----------------------------- Glossary Entries -----------------------------

\newglossaryentry{path}{
    name={path},
    description={A path \(P_{i, j}\) exists between two neurons \(i, j\) if the value of \(i\) is fed into \(j\).},
}

\newglossaryentry{message}{
    name={message},
    description={Messages are defined in \fullref{subsubsect:message-notation}}
}

\end{document}