\RequirePackage{silence}
\WarningFilter{scrbook}{Usage of package `fancyhdr'}  % Hepthesis issues
\WarningFilter{scrbook}{Usage of package `tocbibind'} % Hepthesis issues
\WarningFilter{biblatex}{File }  % Bei Gelegenheit gegenprüfen!
\WarningFilter{tocbasic}{`tocbibind' redefinition of `\listoffigures'}
\WarningFilter{tocbasic}{`tocbibind' redefinition of `\listoftables'} 
\WarningFilter{chktex}{You should perhaps use `\min' instead.} % unfortunately -
\WarningFilter{chktex}{You should perhaps use `\max' instead.} % - doesn't work
\WarningFilter{latex}{Marginpar on page} % due to todonotes
\WarningFilter{LaTeX}{Command }
\documentclass{akdai}

% Language-Encoding und Font
\usepackage{polyglossia}        % Alternative zu Babel
\usepackage{csquotes}           % Required by Polyglossia
\setmainlanguage[variant=british]{english}
\setotherlanguage[babelshorthands=true]{german}

% Citation, Verweise
\usepackage[style=ieee,backend=biber]{biblatex}
\addbibresource{BibLaTex/citation_db.bib}
\usepackage{verbatimbox}
\usepackage{fancyvrb}
\usepackage[style=base]{caption}
\usepackage{listings}           % refer to 'minted vs. texments vs. verbments'
\usepackage{xcolor}
\usepackage{hyperref}           % Verlinkungen im Dokument
\hypersetup{
 colorlinks,
 linkcolor={red!45!black},
 citecolor={blue!50!black},
 urlcolor={blue!80!black}
}
\usepackage{nameref}
\usepackage{booktabs}           % Tabellenpaket-keine vertikal und dicken Linien
\usepackage[inline,shortlabels]{enumitem}
\usepackage[acronym]{glossaries}		% \gls{<golssary-entry>}
\setacronymstyle{long-short}
\usepackage{cleveref}
\lstset{
  frame=single,
  columns=fullflexible,
  literate={-}{-}1
}

\lstnewenvironment{python}[1][]{
  \definecolor{commentsColor}{rgb}{0.497495, 0.497587, 0.497464}
  \definecolor{keywordsColor}{rgb}{0.000000, 0.000000, 0.635294}
  \definecolor{stringColor}{rgb}{0.558215, 0.000000, 0.135316}
  \lstset{
    float=h,
    backgroundcolor=\color{white},                % choose the background color; you must add \usepackage{color} or \usepackage{xcolor}
    basicstyle=\footnotesize,                     % the size of the fonts that are used for the code
    breakatwhitespace=false,                      % sets if automatic breaks should only happen at whitespace
    breaklines=true,                              % sets automatic line breaking
    captionpos=b,                                 % sets the caption-position to bottom
    commentstyle=\color{commentsColor}\textit,    % comment style
    deletekeywords={},                            % if you want to delete keywords from the given language
    escapeinside={\%*}{*)},                       % if you want to add LaTeX within your code
    extendedchars=true,                           % lets you use non-ASCII characters; for 8-bits encodings only, does not work with UTF-8
    frame=tb,	                   	                % adds a frame around the code
    keepspaces=true,                              % keeps spaces in text, useful for keeping indentation of code (possibly needs columns=flexible)
    keywordstyle=\color{keywordsColor}\bfseries,  % keyword style
    language=Python,                              % the language of the code (can be overrided per snippet)
    otherkeywords={},                             % if you want to add more keywords to the set
    numbers=left,                                 % where to put the line-numbers; possible values are (none, left, right)
    numbersep=5pt,                                % how far the line-numbers are from the code
    numberstyle=\tiny\color{commentsColor}\noncopynumber,       % the style that is used for the line-numbers
    rulecolor=\color{black},                      % if not set, the frame-color may be changed on line-breaks within not-black text (e.g. comments (green here))
    showspaces=false,                             % show spaces everywhere adding particular underscores; it overrides 'showstringspaces'
    showstringspaces=false,                       % underline spaces within strings only
    showtabs=false,                               % show tabs within strings adding particular underscores
    stepnumber=1,                                 % the step between two line-numbers. If it's 1, each line will be numbered
    stringstyle=\color{stringColor},              % string literal style
    tabsize=2,                                    % sets default tabsize to 2 spaces
    title=\lstname,                               % show the filename of files included with \lstinputlisting; also try caption instead of title
    columns=fullflexible,                         % Using fixed column width produces nice alignment. I think it's ugly, though -> fullflexible
    literate={-}{-}1
             {*}{*}1
             {\ }{{\copyablespace}}1 {\ \ }{{\copyablespaceTwo}}1,
    #1                                            % Optional arguments
    }  
  }{} 

% Fonts
\usepackage{fontspec}
\usepackage{unicode-math} % fontspec wird auch von unicode-math geladen?
%\setmainfont{Times New Roman} % set fontsize: 11p
%\setmainfont{TeX Gyre Pagella} % set fontsize: 12p
%\setmathfont[ItalicFont=*, BoldFont=*]{TeX Gyre Pagella Math}

% Math-Packages und Fonts
\usepackage{physics} % ||norm|| und |abs|
\usepackage{amsmath}
\usepackage{mathrsfs} % für \mathscr

% Theoreme und Definitionen
\usepackage{amsthm}
\theoremstyle{definition}
\newtheorem{definition}{Definition}[section]

% Formatierung
\usepackage{geometry}           % Paket für Zeilenabstand 
\usepackage{setspace}           % Paket für Zeilenabstand 
\usepackage{microtype}


% Subsections
\setcounter{secnumdepth}{5}
\setcounter{tocdepth}{5}

% Fileinsertion
\usepackage{pdfpages}           % Paket zum Einfügen von PDFs

% Sonstiges                     % Einfügen von Grafiken
\usepackage{graphicx}
\usepackage{subcaption}         % Replacement for subfig package which is broken
\graphicspath{ {./pictures/} }
\usepackage{xcolor}
\usepackage[normalem]{ulem}     % Unterstreichen von Text mit \uline
\usepackage[de-DE]{datetime2}   % Kalenderdaten in deutschem Format
\usepackage{blindtext}          % Repräsentativer als Lorem Ipsum
\usepackage{kantlipsum}         % Cooler als blindtext
\setlength{\marginparwidth}{2cm}
\usepackage[disable,textwidth=15mm]{todonotes} % Einfügen von Todos, Option: [disable]
\usepackage[shortcuts]{extdash} % \=/ for nonbreaking dash
\usepackage{url}
\crefname{const}{constraint}{constraints}
\usepackage{xifthen}
\usepackage{xargs}
\usepackage{array,tabularx}

%%%%% Workarounds %%%%%
\setlength{\marginparwidth}{2cm} % without: trouble with todonotes

\newcommand*\appendixmore{ % add title to Abstract
  \addsec{\appendixname}%
  \renewcommand{\thesubsection}{\Alph{subsection}}%
}

% list of listings title
\renewcommand*{\lstlistlistingname}{List of listings}

% Make Python-Listing line numbers uncopyable
\usepackage[space=true]{accsupp}
\newcommand{\noncopynumber}[1]{%
    \BeginAccSupp{method=escape,ActualText={}}%
    #1%
    \EndAccSupp{}%
}
% Fix spaces in Python listing (really necessary?)
\newcommand{\copyablespace}{\BeginAccSupp{method=hex,unicode,ActualText=00A0}\ \EndAccSupp{}}
\newcommand{\copyablespaceTwo}{\BeginAccSupp{method=hex, unicode, ActualText=00A000A0}\ \ \EndAccSupp{}}
\makeatletter
  \def\lst@Literatekey#1\@nil@{\let\lst@ifxliterate\lst@if
  \expandafter\def\expandafter\lst@literate\expandafter{\lst@literate#1}}
\makeatother

%%%%% /Workarounds %%%%%

\author{
	Malte Zietlow, Finn Wellershaus \\ 
	\\
	Nordakademie\\ 
	Köllner Chaussee 11\\ 
	25337 Elmshorn \\ 
	malte.zietlow@nordakademie.de \\
	finn.wellershaus@nordakademie.de
}
\makeglossaries{}             % makeglossaries <document-root ohne .tex> per cmd

\begin{document}

\title{DeepFakes}
\maketitle

\listoftodos{}            % Aktuelle ToDo-Notes

% Frontmatter
\begin{abstract}
    In this paper we give a theoretical overview of methods for replicating a
    generic content manipulation technique named \textit{DeppFake}. Our focus
    lies on providing a mathematical formulation and giving a technical demo of
    it.
\end{abstract}
% Mainmatter
\section{Introduction}
In late 2017, \gls{redditor} \textit{deepfake} began publishing pornographic footage
on \url{https://www.reddit.com/r/deepfakes}, where he replaced the actors' faces
with those of celebrities --- without their consent. As a consequence, he
received increased media coverage in early 2018, after the release of a first 
article and interview on vice-motherboard~\cite{Cole.2017}. While he coined the
term \textit{DeepFake} for face-\textit{replacement} and face-\textit{reenactment}
technologies, both were already actively researched in academia with research 
dating back to at least 1997~\cite{Bregler.1997}.
Following this, there was an increased awareness for \textit{DeepFakes} which also
caused more research to be conducted, from \(3\) papers published in \(2017\) to
more than \(150\) in \(2018-2019\)~\cite{mirsky_creation_2020} as well as
resulting in first government action~\cite{senate_-_homeland_security_and_governmental_affairs__house_-_energy_and_commerce_deepfake_2019}.

\par
The aim of this paper is
\begin{enumerate*}[a.)]
    \item to give the reader an introduction to the process of creating manipulated
    images via the generic content manipulation technique named \textit{DeepFake}
    and
    \item to give an overview on different approaches to the detection of such manipulated footage.
\end{enumerate*}

\subsection{Structure}
In the remainder of this section, the limitations to the scope of this paper are
presented. Then, in \cref{sect:background}, an overview and definition of \textit{DeepFakes}
is given, including also a discussion of ethical challenges.
This is followed by a summary of the creation of many-to-many \textit{DeepFakes}
in \cref{sect:creation-of-deepfakes}. Finally, different methods of detecting
\textit{DeepFakes} can be found in \cref{sect:detection}.

\subsection{Limitations}\label{subsect:limitations}
While there are multiple \textit{flavors} of \textit{DeepFakes} (see \cref{subsubsect:deepfake-flavors}),
this study explicitly works within the scope of many-to-many mimical puppetry and its
detection. This means that neither \textit{DeepFakes} of text or audio signals,
face-replacement or body reenactment are part of this work. This is mainly
due to spatial constraints and considering that in their core principles the techniques are mostly
similar. Beyond that, one-to-one and one-to-many \textit{DeepFakes} (see \cref{subsubsect:generalizability-deepfakes})
are not examined. This restriction is made because we consider many-to-many \textit{DeepFakes}
to be of greater risk due to their more generic nature and thus larger applicability.
\todo{unify all the sources, no duplications}
\section{Mathematical Formulation}
Mathematical formulation of the problem. With fancy equations, see e.g.\
\cref{eq:simple-eq}.
\begin{equation}\label{eq:simple-eq}
  x=\sqrt{x^2}
\end{equation}
\section{Technical Demo}
Details on the technical demo with a simple python listing, see \cref{lst:python-example}.
\begin{python}[caption={Python Hello-World example},label={lst:python-example},aboveskip={\bigskipamount}]
  print(``ttt'')
\end{python}
\section{Conclusion}
There are different approaches to dealing with \textit{DeepFakes}, some more
promising than others. We consider methods of detection to be only one part of a
possible solution since, on their own, they would possibly only lead to a race
between \glspl{nn} for the creation and for the detection of \textit{DeepFakes},
both getting more and more sophisticated. In this aspect, we agree with~\textcite{Mirsky.2020}'s
ideas that out-of-band methods for signatures of multimedia content and other
prevention mechanisms are required for a better solution. However, a sole focus
on technical solutions might also be futile. More effective might be to, at the
same time, try to adapt on a legal and society level to this new situation,
being aware of the existence of possibly false media content.


% Backmatter
\appendix

%\chapter{General Todos}

% Bibliography and lists 
{\hypersetup{hidelinks}
    \printbibliography{}
    \printglossaries{}
    \listoffigures
    \listoftables
}


% ----------------------------- Acronyms -----------------------------
\newacronym{nn}{NN}{Neural Network}
\newacronym{ai}{AI}{Artificial Intelligence}
\newacronym{cnn}{CNN}{Convolutional Neural Network}
\newacronym{gan}{GAN}{Generative Adversarial Network}

% ----------------------------- Glossary Entries -----------------------------

\newglossaryentry{path}{
    name={path},
    description={A path \(P_{i, j}\) exists between two neurons \(i, j\) if the value of \(i\) is fed into \(j\).},
}

\newglossaryentry{message}{
    name={message},
    description={Messages are defined in \fullref{subsubsect:message-notation}}
}

\end{document}