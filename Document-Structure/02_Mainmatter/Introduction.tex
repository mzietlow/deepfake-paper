\section{Introduction}
In late 2017, \gls{redditor} \textit{deepfake} began publishing pornographic footage
on \url{https://www.reddit.com/r/deepfakes}, where he replaced the actors' faces,
with those of celebrities, without their consent. As a consequence, he
received increased media coverage in early 2018, after the release of a first 
article and interview on vice-motherboard~\cite{Cole.2017}. While he coined the
term \textit{DeepFake} for face-\textit{replacement} and face-\textit{reenactment}
technologies, both were already actively researched in academia with research 
dating back to at least 1997~\cite{Bregler.1997}.
Following this there was an increase awareness for \textit{DeepFakes} which also
caused more reasearch to be conducted, from \(3\) papers pulished in \(2017\) to
more than \(150\) in \(2018-2019\)~\cite{mirsky_creation_2020} as well as
resulting in first government action~\cite{senate_-_homeland_security_and_governmental_affairs_deepfake_2019}.

\par
The aim of this paper is
\begin{enumerate*}[a.)]
    \item to give the reader an intuition of the process of creating manipulated
    images via the generic content manipulation technique named \textit{DeepFake}
    and
    \item to give an overview of approaches to detect such manipulated footage.
\end{enumerate*}

\subsection{Structure}
In the remainder of this section, the limitations to the scope of this paper are
presented. Then in \cref{sect:background}, an overview and definition of \textit{DeepFakes}
is given, including also a discussion of ethical challenges.
This is followed by an overview of the creation of many-to-many \textit{DeepFakes}
in \cref{sect:creation-of-deepfakes}. Finally, different approaches to detecting
\textit{DeepFakes} are given in \cref{sect:detection}.

\subsection{Limitations}\label{subsect:limitations}
While there are multiple \textit{flavors} of \textit{DeepFakes} (see \cref{subsubsect:deepfake-flavors}),
this study explicitly limits its scope to many-to-many mimical puppetry and its
detection. This means that neither \textit{DeepFakes} of text or audio signals,
face-replacement or body reenactment are part of this work. This is done mainly
due to spatial constraints as in their core principles the techniques are mostly
similar. Beyond that, one-to-one and one-to-many \textit{DeepFakes} (see \cref{subsubsect:generalizability-deepfakes})
are not examined. This restriction is made because we consider many-to-many \textit{DeepFakes}
to be of greater risk due to their more generic nature and thus larger applicability.
\todo{unify all the sources, no duplications}