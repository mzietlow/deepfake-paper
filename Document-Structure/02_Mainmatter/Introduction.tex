\section{Introduction}
In late 2017, \gls{redditor} \textit{deepfake} began publishing pornographic footage
on \url{https://www.reddit.com/r/deepfakes}, where he replaced the actors' faces
with those of celebrities --- without either consent. As a consequence, he
received increased media coverage in early 2018, after the release of a first 
article and interview on vice-motherboard~\cite{Cole.2017}. While he coined the
term \textit{DeepFake} for face-\textit{replacement} and face-\textit{reenactment}
technologies, both were already previously known in academia with research 
dating back to at least 1997~\cite{Bregler.1997}.
Following this however, there was an increased awareness for \textit{DeepFakes}
which caused more research to be conducted, from \(3\) papers published in
\(2017\) to more than \(150\) in \(2018-2019\)~\cite{Mirsky.2020} as
well as resulting in first government action~\cite{senate_-_homeland_security_and_governmental_affairs_deepfake_2019}.

\par
The aim of this paper is
\begin{enumerate*}[a.)]
    \item to give the reader an introduction to the process of creating manipulated
    images via deep learning, i.e.\ creating \textit{DeepFakes} and
    \item to give an overview on different approaches to the detection of such
    manipulated footage.
\end{enumerate*}

\subsection{Structure}
In the remainder of this section, the limitations to the scope of this paper are
presented. Then, in \cref{sect:background}, a distinction between \textit{DeepFakes}
and manual video manipulation is made, followed by a discussion of ethical challenges.
This is followed by a summary of the creation of many-to-many \textit{DeepFakes}
in \cref{sect:creation-of-deepfakes}. Finally, different methods for detecting
\textit{DeepFakes} can be found in \cref{sect:detection}.

\subsection{Limitations}\label{subsect:limitations}
While there are multiple \textit{flavors} of \textit{DeepFakes} (see \cref{subsubsect:deepfake-flavors}),
this study explicitly works within the scope of many-to-many face reenactment and its
detection. This means that neither \textit{DeepFakes} of text or audio signals,
face-replacement or body reenactment are part of this work. This is mainly
due to spatial constraints and considering that, in their core principles, the
techniques are mostly equal. Beyond that, one-to-one and one-to-many
\textit{DeepFakes} (see \cref{subsubsect:generalizability-deepfakes}) are not
examined. This restriction is made because we consider many-to-many \textit{DeepFakes}
to be of greater risk due to their more generic nature and thus larger applicability.
\todo{unify all the sources, no duplications}