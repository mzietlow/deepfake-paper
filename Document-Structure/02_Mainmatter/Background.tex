\section{Background}
\subsection{DeepFakes}
In this work we will distinguish between so called CheapFakes and DeepFakes.
CheapFakes are the results of the usage of contemporary video and picture editing software.
DeepFake in contrast describes the results of machine learning algorithms, most 
often some form or combination of neural networks.
The name CheapFake should under no circumstance be interpreted as an assesment of
their effectiveness.
They can still be used to effectively fool users, an example might be a video
of US House Speaker Nancy Peloci which was slowed down for her to appear drunk \todo{add reference}.
The names simply refer to the different levels of sophistication of the different approaches.
Arguably at some point they might even melt together with picture and video editing software
utilizing machine learning more and DeepFake technology becoming more accesscible and integrated
into existing tooling.
\subsection{Detection}
\subsection{Ethical Challenges}
The importance DeepFake-Detection only gets elevated by the ethical challenges 
which arrise from their usage.
The ability to create arbitrary fake images bears multiple risks for missuse:
Firstly there is the aspekt of Missinformation or Propaganda. 
An example would be showing images or videos of political candidates saying or
doing things they have not done.
Another aspect would be Pornography, since it is possible to change faces on 
other people bodies, here lies abig potential for non consentual pornography.
These aspects can be seperated into two main categories, the risk for public or
political figures and the rist for private individuals.
The first category is defined by attackers having more resources at hand for an
attack, e.g.\ an organized action agains a specific political candidate.
The second category in contrast is defined by attackers not having that many
resources, e.g.\ an angry ex-boyfriend.
Even though the second category might be of great risk, since not that much 
might be needed to spread a basic amound of false information, in the end their
capabilites are somewhat limited.
Changes in the ease of use and availablility of DeepFake technologies would 
mostly affect the second category,
since the first already has the necessary resources in the first place.
Depending on this ease of use, most attackers might be expected to fall into the
second category.
These might not use the best technology which potentiall makes them succeptable
for DeepFake Detection methods.
In this respect the DeepFakes created for the first category might pose more of
a challenge.
 
\subsection{Taxonomy}
